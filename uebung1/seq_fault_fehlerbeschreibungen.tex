\documentclass[accentcolor=tud9b]{tudreport}
\usepackage[T1]{fontenc}
\usepackage[utf8]{inputenc}
\usepackage[english]{babel}
\usepackage{graphicx}
\usepackage{amsmath}
\usepackage{url}
\usepackage{hyperref}
\usepackage{listings}

\begin{document}
 \chapter*{Fehlerbeschreibungen zum Programm \texttt{seq\_fault.c}}
 \section*{Kompilierfehler}
 Die beiden Kompilierfehler in \texttt{seq\_fault.c} treten in den Zeilen 19 und 20 auf:
 \begin{lstlisting}
  new_student.matriculation_number = matriculation_number;
  new_student.semester = semester;
 \end{lstlisting}
 Hier wurde der Punktoperator verwendet, um auf die members von \texttt{new\_student} zuzugreifen. Dies würde funktionieren, wenn \texttt{new\_student} ein struct wäre, allerdings handelt es sich hierbei um einen struct pointer. Um auf die members eines struct pointers zuzugreifen, benutzt man den Pfeiloperator (\texttt{->}), sodass die korrigierten Codezeilen lauten:
 \begin{lstlisting}
  new_student->matriculation_number = matriculation_number;
  new_student->semester = semester;
 \end{lstlisting}

 \section*{Laufzeitfehler}
 Der Laufzeitfehler tritt in Zeile 17 auf:
 \begin{lstlisting}
  student* new_student;
 \end{lstlisting}
 Hier wird ein segmentation fault ausgegeben, weil kein Speicherplatz für \texttt{new\_student} allokiert wurde. Um dies zu beheben, wird mit dem Befehl \texttt{malloc} Speicherplatz allokiert, sodass die korrigierte Codezeile lautet:
 \begin{lstlisting}
  student* new_student = malloc(sizeof(student));
 \end{lstlisting}
\end{document}
