\documentclass[accentcolor=tud9b]{tudreport}
\usepackage[T1]{fontenc}
\usepackage[utf8]{inputenc}
\usepackage[german]{babel}
\usepackage{graphicx}
\usepackage{amsmath}
\usepackage{url}
\usepackage{hyperref}
\usepackage{listings}
\usepackage{import}
\usepackage{tikz}
\usepackage{subcaption}
\usepackage{float}

\begin{document}
\chapter{Aufgabe 3: Leistungsmodell für Aufgabe 1}

Extra-P gibt ein analytisches Modell aus mit der Funktion 
\begin{equation}
 f(p) = 37.3996 + 2.17271 p \log_2 (p)
\end{equation}
mit $p$ als Anzahl der Prozesse.
Die Laufzeit befindet sich also in $\mathcal{O}(p\log_2(p))$.

\end{document}
