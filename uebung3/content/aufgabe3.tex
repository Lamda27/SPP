\chapter{Aufgabe 3: Speedup}

\textbf{a)} Den Speed-Up $S$ kann man nach \textsc{Amdahl}s Gesetz berechnen nach
\begin{equation}
 S = \frac{1}{(1-f_\textrm{enhanced}) + \frac{f_\mathrm{enhanced}}{S_\mathrm{enhanced}}},
\end{equation}
wobei $f_\mathrm{enhanced}$ der Bruchteil des Programms ist, der von der Verbesserung profitiert und $S_\mathrm{enhanced}$ der Speed-Up des Bruchteils durch die Verbesserung darstellt.

Im Fall, dass sämtliche Gleitpunktoperationen ($f_\mathrm{enhanced} = 0.6$) um den Faktor $S_\mathrm{enhanced} = 1.5$ beschleunigt werden, erreicht man insgesamt einen Speed-Up von
\begin{equation}
 S = \frac{1}{(1-0.6) + \frac{0.6}{1.5}} = 1.25.
\end{equation}
Im Fall, dass nur die Quadratwurzelberechnung ($f_\mathrm{enhanced} = 0.15$) um den Faktor $S_\mathrm{enhanced} = 8$ beschleunigt wird, erreicht man insgesamt einen Speed-Up von 
\begin{equation}
 S = \frac{1}{(1-0.15) + \frac{0.15}{8}} = 1.511.
\end{equation}
Fazit: Die Anwendung würde am meisten von einer Verbesserung der Quadratwurzelberechnung profitieren.\medskip 

\textbf{b)} \textsc{Amdahl}s Gesetz für Parallelisierung lautet
\begin{equation}
 S = \frac{1}{f_\mathrm{seq} + \frac{1-f_\mathrm{seq}}{p}}
 \label{eq:amdahl_parallel}
\end{equation}
mit $f_\mathrm{seq}$ als sequentiellen Bruchteil der Anwendung und $p$ als Prozessoranzahl. In diesem Fall ergibt sich mit $f_\mathrm{seq} = 1-0.9 = 0.1$ und $p = 16$ ein Speed-Up von
\begin{equation}
 S = \frac{1}{0.1+\frac{1-0.1}{16}} = 6.4.
\end{equation}

\textbf{c)} Es gilt, \eqref{eq:amdahl_parallel} mit $p = 16$ und $S = 10$ nach der unbekannten Größe $f_\mathrm{seq}$ aufzulösen:
\begin{align}
 \frac{1}{f_\mathrm{seq}+\frac{1-\mathrm{seq}}{16}} &= 10, \\ 
 f_\mathrm{seq} + \frac{1-f_\mathrm{seq}}{16} &= \frac{1}{10}, \\
 16 f_\mathrm{seq} + 1 - f_\mathrm{seq} &= \frac{16}{10}, \\
 15 f_\mathrm{seq} &= \frac{16}{10} - 1, \\
 f_\mathrm{seq} &= \frac{\frac{16}{10}-1}{15} \approx 0.04.
\end{align}
Somit ergibt sich der Anteil der Anwendung, der parallelisierbar sein muss, zu $1-f_\mathrm{seq} = 1-0.04 = 0.96$. Also muss $96\,\%$ der Ausführungszeit perfekt parallisierbar sein, um mit 16 Prozessoren einen Speed-Up von 10 zu realisieren.